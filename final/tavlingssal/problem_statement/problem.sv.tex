\problemname{Tävlingssal}
När man anordnar en tävling som PO är det viktigt att se till att deltagarna
inte sitter för nära varandra under tävlingen. På så sätt undviker man att
deltagarna blir störda av andra samtidigt som man motverkar fusk.
Arrangörerna har kommit fram till att deltagarna ska sitta i mönster som ser ut
som ett regelbundet rutnät med avståndet minst 1 till närmaste granne (se bild (TODO: fixa bild)).
Avståndet från en deltagare ut till väggen ska också vara minst 1.
Tävlingssalen ska dessutom vara en
rektangel vars sidor är parallella med rutnätet.

Givet antalet deltagare $N$, bestäm minsta möjliga arean för tävlingssalen, givet att man placerar
deltagarna optimalt.

\section*{Input}
Ett heltal $N$ på en enda rad - antalet deltagare.

\section*{Output}
Skriv ut ett heltal på en enda rad - den minsta möjliga arean för tävlingssalen.

\section*{Förklaring av exempel}
TODO.

\section*{Poängsättning}
Din lösning kommer att testas på en mängd testfallsgrupper. För att få poäng för en grupp så måste du klara alla testfall i gruppen.

\begin{tabular}{| l | l | l | l |}
\hline
Grupp & Poängvärde & Gränser & Övrigt\\ \hline
1     & 40         & $ 1 \le N \le 1000 $ & \\ \hline
2     & 30         & $ 1 \le N \le 10^6 $ & \\ \hline
3     & 30         & $ 1 \le N \le 10^9 $ & \\ \hline
\end{tabular}
