\problemname{Muffinspelet}

Alf och Beata var två ungdomar som levde för länge, länge sedan, på tiden innan man kunde spendera sina eftermiddagar med programmeringstävlingar. Deras liv var således mycket tråkigare än vad dagens ungdomars liv är. Hur man kan överleva utan datorer, kanske du frågar dig. Svaret är enkelt: man bakar!

Våra två ungdomar älskade att baka muffins, och hade ofta stora högar när de var klara med bakningen varje dag. För att inte fylla sina kök med muffins utmanade Beata sin kompis på ett spel varje kväll - \emph{Muffinspelet}.

Muffinspelet spelas av två spelare (i vårt fall, Alf och Beata), samt en stor hög med $N$ muffins. Spelarna turas nu om att göra drag. Ett drag går ut på att spelare $A$ delar upp muffinshögen i två delar (där en av högarna kanske är tom). Motspelaren väljer sedan en av högarna, och äter upp alla muffins i högen. I nästa drag byter spelarna roll, så spelare $B$ delar upp muffinshögen och spelare $A$ äter upp en av högarna. De turas om på detta vis ända tills alla muffins är slut.

Alf börjar med att göra ett drag (dvs att dela upp den stora högen), och Beata börjar med att äta upp en av högarna. Kan du beräkna hur många muffins Alf och Beata kommer äta under spelets gång om de båda spelar så bra som möjligt?

\section*{Indata}
Den första och enda raden i indatan innehåller heltalet $N$, antalet muffins i högen från början.

\section*{Utdata}
Du ska skriva ut två heltal: antalet muffins som Alf kommer äta och antalet muffins som Beata kommer äta om de båda spelar så bra som möjligt.

\section*{Poängsättning}
\begin{itemize}
	\item För att få en poäng måste du klara testfall där $N \le 20$.
	\item För att få två poäng måste du klara testfall där $N \le 10\,000$.
\end{itemize}


\section*{Förklaring exempel 1}
Eftersom det bara finns en muffin är den enda möjliga uppdelningen Alf kan göra en tom hög och en hög med en muffin. Beata kommer då äta upp högen med en muffin. 

\section*{Förklaring exempel 2}
Här kan Alf fortfarande bara få en muffin. Den första rundan delar han upp alla muffin i två högar med två muffins var. Beata äter upp två muffins, och delar sedan upp den kvarvarande högen i två högar med en muffin kvar. Alf äter upp en muffin och måste sedan låta Beata få den sista muffinen.
